% FP-Dokumentenklasse laden
\documentclass[
    abteilunga,         % Abteilung des Versuchs: a, b oder c
    english,ngerman     % Letzte Sprache ist aktive Sprache; EN & DE angeben!
]{tudfp}

%-----------------------------------------------------------------------------
% Allgemeine Informationen zum Protokoll (für Titelseite, PDF-Eigenschaften)
  \VersuchsNr{{9.3}}
  \Titel{{Bestimmung der Belastbarkeit von Physikstudenten mittels Monte-Carlo}}
  \UTitel{{Anderer γ-Titel}}
  
  \Betreuer{{Richard Feynman}}
  \LabDate{2016-02-16} % Durchgeführt am
  \ReleaseDate{2016-02-17} % Abgabe
  
  \AutorA{{Marie Curie}}
  \AutorAMatr{{2828427}}
  \AutorAMail{marie.curie@stud.tu-darmstadt.de}

  \AutorB{{Erwin Schrödinger}}
  \AutorBMatr{{2718281}}
  \AutorBMail{erwin.schrödinger@stud.tu-darmstadt.de}


%-----------------------------------------------------------------------------
% Eigene Pakete/Anpassungen

% \usepackage{graphicx}         % Grafiken einbinden
% \usepackage{tikz}             % Zeichnungen erstellen

% \usepackage{biblatex}         % Bibliograhie
% \addbibresource{bib/literatur.bib}
  
% \usepackage{longtable}
% \usepackage{multirow}
% \usepackage{booktabs}


%-----------------------------------------------------------------------------
% Dokumentenanfang
\begin{document}

% Titelseite
\maketitle

% Inhaltsverzeichnis
\tableofcontents
\cleardoublepage

%-----------------------------------------------------------------------------
% Eigentlicher Protokollinhalt
%-----------------------------------------------------------------------------

\section{Einführung}
%\input{tex/intro.tex}


\section{Theoretische Grundlagen}
%\input{tex/grundlagen.tex}


\section{Versuchsaufbau}
%\input{tex/aufbau.tex}


\section{Durchführung}
%\input{tex/execution.tex}


\section{Auswertung}
%\input{tex/auswertung.tex}


\section{Fazit}
%\input{tex/fazit.tex}


% Literaturverzeichnis
% \nocite{*} % Alle Quellen (auch ohne Zitieren) auflisten
% \printbibliography[heading=bibnumbered]


\section{Messdaten}
%\input{tex/daten.tex}


%-----------------------------------------------------------------------------
\end{document}
